% Kompatibilität mit neuen Zeichensätzen
\usepackage[utf8]{inputenc}
\usepackage{textcomp}

% Bildunterschriften
\usepackage{caption}
\usepackage{subcaption}
% \usepackage[figurename=Abb.]{caption} 

% Graifkimport
\usepackage{graphicx}
\usepackage{pdfpages}

% Blatteigenschaften
\usepackage{geometry}
% \geometry{a4paper, left=25mm, right=25mm, top=25mm, bottom=25mm}
\usepackage[onehalfspacing]{setspace} % Zeilenabstand 1.5

% keine Einrückung
% \setlength{\parindent}{0pt}
% \usepackage{floatrow}

% Mathematikpakete
\usepackage{amsmath}
\usepackage{upgreek}
\usepackage{listings}
\usepackage{siunitx}

% Zeichenprogramme
\usepackage{tikz}
\usepackage{quantikz}

% Bibliographie
\usepackage[citestyle=authoryear, style=phys, backend=biber]{biblatex}
\usepackage{csquotes}
\addbibresource{bibliography.bib}

% Referenzierung
\usepackage{hyperref}
\hypersetup{
            colorlinks=false,
            pdfborder=0 0 0,
            pdfauthor   = {Maximilian Konrad},
	        pdftitle    = {Quantum Computing},
	        pdfsubject  = {Bachelorarbeit},
	        pdfkeywords = {QuantumComputing, Bachelorarbeit}}
\urlstyle{same}

% Todos
\usepackage{todonotes}

% Footnote
\usepackage{perpage}
\MakePerPage{footnote}
\usepackage[symbol]{footmisc}
\setcounter{footnote}{2}
\newcommand{\myfootnote}{\fnsymbol{footnote}}

% Abstände vor und nach Formeln
\setlength\abovedisplayshortskip{10pt}
\setlength\belowdisplayshortskip{10pt}
\setlength\abovedisplayskip{10pt}
\setlength\belowdisplayskip{10pt}

% Mathematische Formulierungen
\renewcommand{\Im}{\operatorname{Im}}
\renewcommand{\Re}{\operatorname{Re}}
\newcommand{\abs}[1]{\left| #1 \right|}

% Silbentrennung
% \hyphenation{haupt-söch-lich}

%%% Local Variables:
%%% mode: latex
%%% TeX-master: "../MainDocument"
%%% End:
